\documentclass{beamer}
\usepackage[latin1]{inputenc}
\usepackage{listings}
\usepackage{color}
\usepackage{ulem}

\newcommand {\framedgraphic}[2] {
    \begin{frame}{#1}
        \begin{center}
            \includegraphics[width=\textwidth,height=0.8\textheight,keepaspectratio]{#2}
        \end{center}
    \end{frame}
}

\usetheme{Warsaw}
\setbeamertemplate{footline}{\insertframenumber/\inserttotalframenumber}

\title[Gibbs Sampling]{Gibbs Sampling \\ Some Theory and Some Practice}
\author{Dominic Steinitz}
\date{25th June 2014}
\begin{document}

\begin{frame}
\titlepage
\end{frame}

\section{Introduction}

\subsection{What This Talk is About}

\begin{frame}{What This Talk Is About}

  \begin{itemize}
  \item Introduction to Markov chain methods.
  \item Two Gibbs samplers: theory.
  \item .
  \item .
  \end{itemize}

\end{frame}

\section{Introduction to Markov Chain Methods}

\subsection{The Grasshpper's Problems}

\framedgraphic{Visit Each Hour Equally}{diagrams/ClockEqual12.png}

\begin{frame}[fragile]{Suppose Achieve Goal}

  \uncover<1->{
    $$
    \pi'(n) = p(n \, |\, n - 1)\pi(n - 1) + p(n \, |\, n + 1)\pi(n + 1)
    $$
  }
  \uncover<2->{
    $$
    \pi'(n) = \frac{1}{2}\frac{1}{N} + \frac{1}{2}\frac{1}{N} = \frac{1}{N}
    $$
  }

\end{frame}

\begin{frame}[fragile]{Testing Convergence}
\begin{verbatim}
ghci> startOnOne
  (1><5)
   [ 1.0, 0.0, 0.0, 0.0, 0.0 ]

ghci> eqProbsMat
  (5><5)
   [ 0.0, 0.5, 0.0, 0.0, 0.5
   , 0.5, 0.0, 0.5, 0.0, 0.0
   , 0.0, 0.5, 0.0, 0.5, 0.0
   , 0.0, 0.0, 0.5, 0.0, 0.5
   , 0.5, 0.0, 0.0, 0.5, 0.0 ]
\end{verbatim}
\end{frame}

\begin{frame}[fragile]{Testing Convergence}
\begin{verbatim}

ghci> take 1 $
      drop 1000 $
      iterate (<> eqProbsMat) startOnOne
  [(1><5)
   [ 0.2, 0.2, 0.2, 0.2, 0.2 ]]
\end{verbatim}
\end{frame}

\framedgraphic{Visit Each Hour Proportionally}{diagrams/ClockProp12.png}

\begin{frame}[fragile]{Suppose Achieve Goal}
  \uncover<1->{
    $$
    \pi'(n) = p(n \, |\, n - 1)\pi(n - 1) + p(n \, |\, n)\pi(n) + p(n \, |\, n + 1)\pi(n + 1)
    $$
  }
  \uncover<2->{
    $$
    \begin{aligned}
      \pi'(4) &= \frac{1}{2}\pi(3) +
      \frac{1}{2}\frac{1}{4}\pi(4) +
      \frac{1}{2}\frac{4}{5}\pi(5) \\
      &= \frac{1}{2}\bigg(\frac{3}{N} + \frac{1}{4}\frac{4}{N} + \frac{4}{5}\frac{5}{N}\bigg) \\
      &= \frac{1}{N}\frac{8}{2} \\
      &= \frac{4}{N} \\
      &= \pi(4)
    \end{aligned}
    $$
  }
\end{frame}

\begin{frame}[fragile]{Testing Convergence}
\begin{verbatim}
    ghci> incProbsMat
    (5><5)
    [ 0.00, 0.500, 0.000, 0.000, 0.5
    , 0.25, 0.250, 0.500, 0.000, 0.0
    , 0.00, 0.333, 0.167, 0.500, 0.0
    , 0.00, 0.000, 0.375, 0.125, 0.5
    , 0.10, 0.000, 0.000, 0.400, 0.5 ]

    ghci> take 1 $
          drop 1000 $
          iterate (<> incProbsMat) startOnOne
    [(1><5)
      [ 6.67e-2, 0.133, 0.199, 0.267, 0.333 ]]
\end{verbatim}
\end{frame}

\begin{frame}[fragile]{Very Simple Markov Chain}
$$
q(i, j) =
\begin{cases}
\mathbb{P} (X_{n+1} = j \,|\, X_n = i) = \frac{1}{2} & \text{if } j = i + 1 \mod N \\
\mathbb{P} (X_{n+1} = j \,|\, X_n = i) = \frac{1}{2} & \text{if } j = i - 1 \mod N \\
\mathbb{P} (X_{n+1} = j \,|\, X_n = i) = 0 & \text{otherwise }
\end{cases}
$$
\end{frame}

\begin{frame}[fragile]{Very Simple Markov Chain}
  \uncover<1->{
    The grasshopper knows that $\pi(i) = i/N$ so it can calculate
    $\pi(j)/\pi(i) = j/i$ without knowing $N$.
  }
\uncover<2->{
  $$
  p(i, j) =
  \begin{cases}
    q(i,j)\bigg[\frac{\pi(j) q(j,i)}{\pi(i) q(i,j)} \land 1 \bigg] & \text{if } j \ne i \\
    1 - \sum_{k : k \ne i} q(i,k) \bigg[\frac{\pi(k) q(k,i)}{\pi(i) q(i,k)} \land 1 \bigg] & \text{if } j = i
  \end{cases}
  $$
}
\end{frame}

\section{Differentiation Methods}

\subsection{Bumping}

\begin{frame}{Bumping}
Could try bumping
$$
\frac{\partial L(\ldots, w, \ldots)}{\partial w} \approx \frac{L(\ldots, w + \epsilon, \ldots) - L(\ldots, w, \ldots)}{\epsilon}
$$
\begin{itemize}
\item
But we have many thousands of parameters
\item
And we are using floating point arithmetic \ldots
\end{itemize}
\end{frame}

\framedgraphic{Floating Point Errors}{diagrams/13a2bd186a0e123f040da9491fa98684.png}

\subsection{Symbolic Differentiation}

\begin{frame}[fragile]{Python Example}
  Could use a symbolic differentiation package but consider the
  following Python fragment
\begin{lstlisting}[language=Python]
import numpy as np

def many_sines(x):
    y = x
    for i in range(1,7):
        y = np.sin(x+y)
    return y
\end{lstlisting}
\end{frame}

\begin{frame}[fragile]{Python Example}
When we unroll the loop we are actually evaluating

$$
f(x) = \sin(x + \sin(x + \sin(x + \sin(x + \sin(x + \sin(x + x))))))
$$

\end{frame}

\begin{frame}[fragile]{Blow Up}
Now suppose we want to get the differential of this
function. Symbolically this would be

$$
\begin{aligned}
f'(x) &=           (((((2\cdot \cos(2x)+1)\cdot \\
      &\phantom{=} \cos(\sin(2x)+x)+1)\cdot \\
      &\phantom{=} \cos(\sin(\sin(2x)+x)+x)+1)\cdot \\
      &\phantom{=} \cos(\sin(\sin(\sin(2x)+x)+x)+x)+1)\cdot \\
      &\phantom{=} \cos(\sin(\sin(\sin(\sin(2x)+x)+x)+x)+x)+1)\cdot \\
      &\phantom{=} \cos(\sin(\sin(\sin(\sin(\sin(2x)+x)+x)+x)+x)+x)
\end{aligned}
$$
\end{frame}

\begin{frame}[fragile]{Moreover}
\begin{itemize}
\item The mathematical function for an MLP could easily be more
  complicated.
\pause
\item To use a symbolic differentiation package (usually) requires
  some manipulation of the original program.
\end{itemize}
\end{frame}

\subsection{Backpropagation}

\begin{frame}[fragile]{Backpropagation}
\begin{itemize}
\item Efficient method for finding the derivative for MLPs (1960's).
\pause
\item Complex but quite often appears in Machine Learning 101.
\pause
\item But maybe following P\'{o}lya we can find a problem that is more
  general and use this not just for MLPs but for estimating parameters
  for other Machine Learning models.
\pause
\item Idea: we drew the Neural Net as a graph maybe if we can draw a
  program as a graph we can calculate the derivative of the function
  which the program represents.
\end{itemize}
\end{frame}

\section{Automatic Differentiation}

\subsection{Reverse Mode}

\begin{frame}[fragile]{A Simpler Example}
Instead of an MLP, consider the function

$$
f(x) = \exp(\exp(x) + (\exp(x))^2) + \sin(\exp(x) + (\exp(x))^2)
$$

We wish to find

$$
\frac{\mathrm{d} f}{\mathrm{d} x}
$$

We can write this function as a data flow graph...
\end{frame}

\framedgraphic{Data Flow Graph}{diagrams/Fita3.png}

\begin{frame}[fragile]{The Chain Rule}
\uncover<1->{
Recall the {\color{brown}{chain rule}}: if $f(x) = g(h(x))$ and  $y =
h(x)$ then
$$
\frac{\mathrm{d} f}{\mathrm{d} x} =
\frac{\mathrm{d} g}{\mathrm{d} y}\frac{\mathrm{d} h}{\mathrm{d} x}
$$
}
\uncover<2->{
Abusing(?) notation
$$
\frac{\mathrm{d} f}{\mathrm{d} x} =
\frac{\mathrm{d} f}{\mathrm{d} y}\frac{\mathrm{d} y}{\mathrm{d} x}
$$
}
\end{frame}

\begin{frame}[fragile]{The Multivariate Chain Rule}
  \uncover<1->{

    We also need the multivariate form: if $f(u) = g(h_1(u),h_2(u))$
    and $x = h_1(u)$ and $y = h_2(u)$ then

$$
\begin{aligned}
\frac{\partial f}{\partial u} &=
 \frac{\partial g}{\partial x}\frac{\partial h_1}{\partial u} +
 \frac{\partial g}{\partial y}\frac{\partial h_2}{\partial u}
\end{aligned}
$$
}

\uncover<2->{
Again, abusing(?) notation

$$
\begin{aligned}
\frac{\partial f}{\partial u} &=
 \frac{\partial f}{\partial x}\frac{\partial x}{\partial u} +
 \frac{\partial f}{\partial y}\frac{\partial y}{\partial u}
\end{aligned}
$$
}
\end{frame}

\framedgraphic{Chain Rule Example: $u_6$ and $u_5$}{diagrams/Fita3.png}

\begin{frame}[fragile]{Chain Rule Example: $u_6$ and $u_5$}
\uncover<1->{
Since
$$
\begin{aligned}
u_7 = u_6 + u_5
\end{aligned}
$$
}
\uncover<2->{
Then
$$
\begin{aligned}
\frac{\partial u_7}{\partial u_6} &= 1 \\
\frac{\partial u_7}{\partial u_5} &= 1
\end{aligned}
$$
}
\end{frame}

\framedgraphic{Chain Rule Example: $u_4$}{diagrams/Fita3.png}

\begin{frame}[fragile]{Chain Rule Example: $u_4$}
\uncover<1->{
Now things become more interesting as $u_7$ depends on $u_4$ via $u_6$
and $u_5$: $u_7 = u_7(u_6(u_4), u_5(u_4))$
}

\uncover<2->{
By the multivariate {\color{brown}{chain rule}}
$$
\frac{\partial u_7}{\partial u_4} =
 \frac{\partial u_7}{\partial u_6}\frac{\partial u_6}{\partial u_4} +
 \frac{\partial u_7}{\partial u_5}\frac{\partial u_5}{\partial u_4}
$$
}
\uncover<3->{
$$
= \frac{\partial u_7}{\partial u_6}\exp{u_4} +
  \frac{\partial u_7}{\partial u_5}\cos{u_4}
$$
}
\end{frame}

\framedgraphic{Chain Rule Example: $u_3$}{diagrams/Fita3.png}

\begin{frame}[fragile]{Chain Rule Example: $u_3$}
\uncover<1->{
$u_7$ depends on $u_3$ only via $u_4$: $u_7 = u_7(u_4(u_3))$
}

\uncover<2->{
By the {\color{brown}{chain rule}}
$$
\begin{aligned}
 \frac{\partial u_7}{\partial u_3} &=
 \frac{\partial u_7}{\partial u_4}\frac{\partial u_4}{\partial u_3} \\
\end{aligned}
$$
}

\uncover<3->{
$$
= \frac{\partial u_7}{\partial u_4}
$$
}
\end{frame}

\framedgraphic{Chain Rule Example: $u_2$}{diagrams/Fita3.png}

\begin{frame}[fragile]{Chain Rule Example: $u_2$}
\uncover<1->{
Now things become more interesting again as $u_7$ depends on $u_2$ via $u_4$
and $u_3$: $u_7 = u_7(u_4(u_2), u_3(u_2))$
}

\uncover<2->{
By the {\color{brown}{chain rule}}
$$
\frac{\partial u_7}{\partial u_2} =
 \frac{\partial u_7}{\partial u_4}\frac{\partial u_4}{\partial u_2} +
 \frac{\partial u_7}{\partial u_3}\frac{\partial u_3}{\partial u_2}
$$
}
\uncover<3->{
$$
= \frac{\partial u_7}{\partial u_4} +
  \frac{\partial u_7}{\partial u_3}2u_2
$$
}
\end{frame}

\framedgraphic{Chain Rule Example: $u_1$}{diagrams/Fita3.png}

\begin{frame}[fragile]{Chain Rule Example: $u_1$}
\uncover<1->{
Finally, $u_7$ depends on $u_1$ via $u_2$: $u_7 = u_7(u_2(u_1))$
}

\uncover<2->{
By the {\color{brown}{chain rule}}
$$
\frac{\partial u_7}{\partial u_1} =
\frac{\partial u_7}{\partial u_2}\frac{\partial u_2}{\partial u_1}
$$
}
\uncover<3->{
$$
= \frac{\partial u_7}{\partial u_2}\exp{u_2}
$$
}
\end{frame}

\begin{frame}[fragile]{Reverse Mode Summary}

\begin{itemize}
\pause
\item Calculate the function and the intermediate values working from
  left to right (the forward sweep)
\pause
\item And calculate the one step derivatives of the
  primitive functions at the same time
\pause
\item Then calculate the required derivative working from right to
  left (the backward sweep)
\pause
\item Same complexity and no loss in precision
\pause
\item Pictorial representation of MLP is the data flow graph of its cost
  function and reverse mode AD is identical to backpropagation
\end{itemize}

\end{frame}

\subsection{Forward Mode}

\begin{frame}[fragile]{Where's the Fun?}
What has this got to do with functional programming?
\end{frame}

\begin{frame}[fragile]{Dual Numbers}
\begin{itemize}
\item
An alternative method for automatic differentiation is called forward
mode and has a simple implementation. Let us illustrate this using
Haskell. The actual implementation is about 20 lines of code.
\pause
\item
Let us define dual numbers

\begin{lstlisting}[language=Haskell]
data Dual = Dual Double Double
  deriving (Eq, Show)
\end{lstlisting}
\pause
\item
We can think of these pairs as first order polynomials in the
indeterminate $\epsilon$, $x + \epsilon x'$ such that $\epsilon^2 = 0$
(Galois theory)
\end{itemize}
\end{frame}

\begin{frame}[fragile]{Dual Numbers}
Thus, for example, we have

\begin{itemize}
\pause
\item $(x + \epsilon x') + (y + \epsilon y') = ((x + y) + \epsilon (x' + y'))$
\pause
\item $(x + \epsilon x')(y + \epsilon y') = xy + \epsilon (xy' + x'y)$
\pause
\item $\log (x + \epsilon x') = \log x (1 + \epsilon \frac {x'}{x}) =
  \log x + \epsilon\frac{x'}{x}$
\pause
\item $\sqrt{(x + \epsilon x')} = \sqrt{x(1 + \epsilon\frac{x'}{x})} =
  \sqrt{x}(1 + \epsilon\frac{1}{2}\frac{x'}{x}) = \sqrt{x} +
  \epsilon\frac{1}{2}\frac{x'}{\sqrt{x}}$
\end{itemize}
\end{frame}

\begin{frame}[fragile]{Dual Numbers and The Chain Rule}
\uncover<1->{
Notice that these equations implicitly encode the chain rule. For
example, we know, using the chain rule, that

$$
\frac{\mathrm{d}}{\mathrm{d} x}\log(\sqrt x) =
\frac{1}{\sqrt x}\frac{1}{2}x^{-1/2} =
\frac{1}{2x}
$$
}

\uncover<2->{
And using the example equations above we have

$$
\begin{aligned}
\log(\sqrt {x + \epsilon x'}) &= \log (\sqrt{x} +
\epsilon\frac{1}{2}\frac{x'}{\sqrt{x}}) \\
&= \log (\sqrt{x}) +
\epsilon\frac{\frac{1}{2}\frac{x'}{\sqrt{x}}}{\sqrt{x}} = \log
(\sqrt{x}) + \epsilon x'\frac{1}{2x}
\end{aligned}
$$
}
\end{frame}

\begin{frame}[fragile]{Dual Numbers Example}
Notice that dual numbers carry around the calculation and the
derivative of the calculation.

\begin{itemize}
\pause
\item
To actually evaluate $\log(\sqrt{x})$
at a particular value, say 2, we plug in 2 for $x$ and 1 for $x'$
$$
\log (\sqrt{2 + \epsilon 1}) = \log(\sqrt{2}) + \epsilon\frac{1}{4}
$$
\pause
\item
Thus the derivative of $\log(\sqrt{x})$ at 2 is $1/4$.
\end{itemize}
\end{frame}

\begin{frame}[fragile]{Dual Numbers are Numbers}
\begin{scriptsize}
\begin{lstlisting}[language=Haskell]
instance Num Dual where
  fromInteger n             = undefined
  (Dual x x') + (Dual y y') = Dual (x + y) (x' + y')
  (Dual x x') * (Dual y y') = Dual (x * y) (x * y' + y * x')
  negate (Dual x x')        = Dual (negate x) (negate x')
  signum _                  = undefined
  abs _                     = undefined
\end{lstlisting}
\end{scriptsize}
\end{frame}

\begin{frame}[fragile]{Dual Numbers Can Be Divided}
\begin{scriptsize}
\begin{lstlisting}[language=Haskell]
instance Fractional Dual where
  fromRational p    = undefined
  recip (Dual x x') = Dual (1.0 / x) (-x' / (x * x))
\end{lstlisting}
\end{scriptsize}
\end{frame}

\begin{frame}[fragile]{Dual Numbers are Quasi Floating Point}
\begin{scriptsize}
\begin{lstlisting}[language=Haskell]
instance Floating Dual where
  pi                = Dual pi        0
  exp   (Dual x x') = Dual (exp x)   (x' * exp x)
\end{lstlisting}
\color{blue}
\begin{lstlisting}[language=Haskell]
  log   (Dual x x') = Dual (log x)   (x' / x)
  sqrt  (Dual x x') = Dual (sqrt x)  (x' / (2 * sqrt x))
\end{lstlisting}
\color{black}
\begin{lstlisting}[language=Haskell]
  sin   (Dual x x') = Dual (sin x)   (x' * cos x)
  cos   (Dual x x') = Dual (cos x)   (x' * (- sin x))
  sinh  (Dual x x') = Dual (sinh x)  (x' * cosh x)
  cosh  (Dual x x') = Dual (cosh x)  (x' * sinh x)
  asin  (Dual x x') = Dual (asin x)  (x' / sqrt (1 - x*x))
  acos  (Dual x x') = Dual (acos x)  (x' / (-sqrt (1 - x*x)))
  atan  (Dual x x') = Dual (atan x)  (x' / (1 + x*x))
  asinh (Dual x x') = Dual (asinh x) (x' / sqrt (1 + x*x))
  acosh (Dual x x') = Dual (acosh x) (x' / (sqrt (x*x - 1)))
  atanh (Dual x x') = Dual (atanh x) (x' / (1 - x*x))
\end{lstlisting}
\end{scriptsize}
\end{frame}


\begin{frame}[fragile]{Example}
That's all we need to do --- let's try it

\begin{scriptsize}
\begin{lstlisting}[language=Haskell]
f = log . sqrt
\end{lstlisting}
\end{scriptsize}

\begin{scriptsize}
\begin{lstlisting}
ghci> f (Dual 2 1)
  Dual 0.347 0.250
\end{lstlisting}
\end{scriptsize}
\end{frame}

\begin{frame}[fragile]{Automatic Differentiation is {\em Not} Symbolic
    Differentiation}
To see that we are *not* doing symbolic differentiation (it's easy to
see we are not doing numerical differentiation) let us step
through the actual evaluation.

$$
\begin{aligned}
f (\mathrm{Dual}\,2\,1) &\longrightarrow (\mathrm{log} \cdot \mathrm{sqrt}) (\mathrm{Dual}\,2\,1) \\
&\longrightarrow \mathrm{log} \bigg(\mathrm{Dual}\,\sqrt{2}\,\frac{1}{2\sqrt{2}}\bigg) \\
&\longrightarrow  \mathrm{log} (\mathrm{Dual}\,1.414\,0.354) \\
&\longrightarrow \mathrm{Dual}\,\log{1.414}\,\frac{0.354}{1.414} \\
&\longrightarrow \mathrm{Dual}\,0.347\,0.250\\
\end{aligned}
$$
\end{frame}

\section{Application}

\begin{frame}[fragile]{Linear Regression}
\begin{itemize}
\item
$$
L(\boldsymbol{x}, \boldsymbol{y}, m, c) = \frac{1}{2n}\sum_{i=1}^n (y_i - (mx_i + c))^2
$$

\pause
\item
\begin{scriptsize}
\begin{lstlisting}[language=Haskell]
cost m c xs ys = (sum (zipWith errSq xs ys)) /
                 (2 * (fromIntegral (length xs)))
  where
    errSq x y = z * z
      where
        z = y - (m * x + c)
\end{lstlisting}
\end{scriptsize}
\end{itemize}
\end{frame}

\begin{frame}[fragile]{Linear Regression}
\begin{itemize}
\item
\begin{scriptsize}
\begin{lstlisting}[language=Haskell]
g m c = cost m c xs ys
\end{lstlisting}
\end{scriptsize}

\pause
\item
\begin{scriptsize}
\begin{lstlisting}[language=Haskell]
zs = (0.1, 0.1) : map f zs
  where

    deriv (Dual _ x') = x'

    f (c, m) = (c - gamma * cDeriv, m - gamma * mDeriv)
      where
        cDeriv = deriv (g (Dual m 0) (Dual c 1))
        mDeriv = deriv (flip g (Dual c 0) (Dual m 1))
\end{lstlisting}
\end{scriptsize}
\end{itemize}
\end{frame}

\begin{frame}[fragile]{Linear Regression}
\begin{scriptsize}
\begin{lstlisting}
ghci> take 10 zs
  [(0.1,0.1),
   (0.554,3.2240000000000006),
   (0.30255999999999983,1.4371599999999995),
   (0.4542824,2.4573704000000003),
   (0.3754896159999999,1.8730778559999997),
   (0.42839290304,2.2059302422400004),
   (0.4059525336255999,2.0145512305216),
   (0.42651316156582386,2.1228327781207037),
   (0.42242942391663607,2.059837404270339),
   (0.43236801802049607,2.0947533284323567)]
\end{lstlisting}
\end{scriptsize}
\end{frame}

\section{Concluding Thoughts}

\begin{frame}[fragile]{Efficiency}
Perhaps AD is underused because of efficiency?

\begin{itemize}
\pause
\item
It seems that the Financial Services industry is aware that AD is {\em
  more} efficient than current practice. Order of magnitude
improvements have been reported.
\pause
\item
Smoking Adjoints: fast evaluation of Greeks in Monte Carlo Calculations
\pause
\item
Adjoints and automatic (algorithmic) differentiation in computational finance
\pause
\item
Perhaps AD is slowly permeating into Machine Learning as well but
there seem to be no easy to find benchmarks.
\pause
\item
Good Morning, Economics Blog
\pause
\item
Andrew Gelman's Blog
\end{itemize}
\end{frame}

\begin{frame}[fragile]{Resources}
\begin{itemize}
\item \textcolor{blue}{http://idontgetoutmuch.wordpress.com}
\item
  \textcolor{blue}{http://en.wikipedia.org/wiki/Automatic\_differentiation}
\item \textcolor{blue}{http://www.autodiff.org}
\item \textcolor{blue}{http://hackage.haskell.org/package/ad}
\item \textcolor{blue}{http://mc-stan.org}
\item \textcolor{blue}{The Monad.Reader Issue 21}
\end{itemize}
\end{frame}

\end{document}
